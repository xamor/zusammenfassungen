\documentclass[a4paper]{scrartcl}
\usepackage[T1]{fontenc}
\usepackage[utf8]{inputenc}
\usepackage[ngerman]{babel}
\usepackage{graphicx}
\usepackage{enumitem}
\usepackage{bbm}
\usepackage{listings}
\usepackage{fancyhdr}
\usepackage{tikz}
\usepackage{ulem}
\pagestyle{fancy}
\usepackage{amssymb}
\usepackage{latexsym}
\fancyhf{}
\begin{document}
\author{Maximilian Ortwein}
\title{CloudDB Zusammenfassung }
%Fußzeile mittig (Seitennummer)
\fancyfoot[C]{\thepage}
%Linie unten
\renewcommand{\footrulewidth}{0.5pt}
\fancyhead[L]{\small{\textbf{CloudDB Zusammenfassung}}}
\fancyhead[R]{\small{Maximilian Ortwein}}
\renewcommand{\headrulewidth}{0.5pt}
\maketitle
\tableofcontents
\pagebreak

\section{Service Based Computing}
\begin{itemize}
\item SaaS - Software as a Service
\item PaaS - Platform as a Service
\item IaaS - Infrastructure as a Service
\item Cloudcomputing bezeichnet SaaS und die CLoud selber, also Software und Hardware
\item Pay-as-you-go service, Amazon z.B. nur das bezahlen was man wirklich an Leistung braucht
\item Private Clouds, Verteilte systeme
\item Cloud Computing ist 5 bis 7x günstiger als Herkömmliches hosting
\end{itemize}
\subsection{Challange and Opportunities}
\begin{itemize}
\item Enterprise und Akademische Anwendungen eignen sich besonders gut für die Cloud. (Data Management)
\item Daten intensive Aufgaben wie OLAP, OLTP und Batch Processing eignen sich sehr gut für die Cloud
\item Altes model: query the world, neues model: download the world
\item Vorteile von Skalierbaren Datenbanken bei: vielen kleinen Anfragen, Rechenintensiven Anfragen und die Daten sind verteilt.
\item Replikation: Daten sind dort wo sie gebraucht werden, Daten sind hochverfügbar
\item Horizontale Partitionierung: Tupels Werden auf Knoten verteilt
\item Vertikale: Spalten werden auf Knoten verteilt
\item 
\end{itemize}



\end{document}